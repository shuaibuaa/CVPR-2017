\documentclass{article}

\usepackage{geometry}
\usepackage{times}
\usepackage{epsfig}
\usepackage{graphicx}
\usepackage{amsmath}
\usepackage{amssymb}
\usepackage{bm}
\usepackage{mathrsfs}
\usepackage{amsfonts}
\usepackage{color}

\setcounter{page}{1}
\setlength{\parindent}{0pt}
\begin{document}

\title{Kernelization on Neural Network}
\author{Shuai Zhang\and Yingchun Zhang}
\maketitle

%-------------------------------------------------------------------------
\section{Introduction}

\subsection{Related Works}
Wilson et al. \cite{DBLP:journals/corr/WilsonHSX15} combined the non-parametric flexibility of kernel methods with the structural properties of deep neural networks.

%-------------------------------------------------------------------------
\section{Methods}

\subsection{Neural Network}

In a two-layer network, the feed-forward network function is

\begin{equation}
\mathbf{y}_k(\mathbf{x}, \mathbf{w}) = h^{(2)}\left(\sum_{j=0}^{M}w_{kj}^{(2)}h^{(1)}\left(\sum_{i=0}^{D}w_{ji}^{(1)}x_i\right)\right).
\end{equation}

Given a training set with input vectors $\{\mathbf{x}_n\}$, where $n = 1, ..., N$, together with a correspounding set of target vectors $\{\mathbf{t}_n\}$, we minimize the error function

\begin{equation}
E(\mathbf{w}) = \frac{1}{2} \sum_{n=1}^{N} {\|\mathbf{y}(\mathbf{x}_n, \mathbf{w}) - \mathbf{t}_n\|}^2.
\end{equation}




%-------------------------------------------------------------------------
\subsection{Kernelized Neural Network}

%----------------------------------@ July 24---------------------------------------

\newpage
\LARGE Progress @ July 24\normalsize\vspace{3ex}

\subsubsection{Nonparametric representation}

\par \textbf{Single-layer neural network}\\

\begin{figure}[h]
\centering
\includegraphics[height=7cm]{Non-para_single-layer}
\end{figure}

\par In a single-layer neural network, the feed-forward network function is 
\begin{equation}
y_n = h \left( \sum\limits_{j=1}^{K} w_{j} \left( \sum\limits_{i=1}^{N} \alpha_{ji} k( \mathbf{x}_{i} , \mathbf{x}_{n}) \right)\right)
\end{equation} 

where $ \{ \mathbf{x}_{i}\}_{i=1}^{N}$ is the unique sample which is a $D$-dimension vector of the dataset, $k$ is a kernel function which builds a new inner product space, and we try to get k linear combination of with the weight $\alpha_{ji}$, then we make a linear combination of the new base with the weight $ w_{j}$. $h$ is an activation function and so as follows.  
All the $w, h, \alpha, k$ and $\mathbf{x}$ shown in the following part have the same meaning as the former ones.\\



\par \textbf{Multi-layer neural network} \\

\begin{figure}[h]
\centering
\includegraphics[height=7cm]{Non-para_muti-layer}
\end{figure}

\begin{equation}
y_n = h^{(2)} \left( \sum\limits_{j=1}^{K} w_{j}^{(2)} \left( \sum\limits_{i=1}^{N} \alpha_{ji}^{(2)} k_2( \mathbf{x}_{i},\left[               
  \begin{array}{c}   
    h^{(1)} \left( \sum\limits_{j=1}^{K} w_{1j}^{(1)} \left( \sum\limits_{i=1}^{N} \alpha_{ji}^{(1)}k_1( \mathbf{x}_{i},\mathbf{x}_n) \right) \right)\\  
    .\\  
    .\\  
    .\\
    h^{(1)} \left( \sum\limits_{j=1}^{K} w_{Mj}^{(1)} \left( \sum\limits_{i=1}^{N} \alpha_{ji}^{(1)}k_1( \mathbf{x}_{i},\mathbf{x}_n) \right) \right)\\
  \end{array}
\right] )
 \right) \right)
\end{equation} 

\par The number in superscript of the symbols represents the layer of neural network, such as, $w_j^{(2)}$ means the weight of the second layer of neural network and so as follows. \\



\subsubsection{Parametric representation}


\par \textbf{Single-layer neural network}\\

\begin{figure}[h]
\centering
\includegraphics[height=7cm]{Para_single-layer}
\end{figure}

\begin{equation}
y_n = h \left( \sum\limits_{j=1}^K w_j k( \bm{\alpha}_j, \mathbf{x}_{n}) \right)
\end{equation}
\par $\bm{\alpha}_j$ is the center of kernel which is a D-dimension vector.\\





\par \textbf{Multi-layer neural network}\\

\begin{figure}[h]
\centering
\includegraphics[height=7cm]{Para_muti-layer}
\end{figure}

\begin{equation}
y_n = h^{(2)} \left( \sum\limits_{j=1}^K w_j^{(2)} k_2( \bm{\alpha}_j^{(2)}, \left[ \begin{array}{c}
	h^{(1)} \left( \sum\limits_{i=1}^K w_{1i}^{(1)} k_1(\bm{\alpha}_i^{(1)},\mathbf{x}_n) \right)\\
	.\\
	.\\
	.\\
	h^{(1)} \left( \sum\limits_{i=1}^K w_{Mi}^{(1)} k_1(\bm{\alpha}_i^{(1)},\mathbf{x}_n) \right)\\
\end{array}
\right])
\right) 
\end{equation}



%----------------------------------@ July 26---------------------------------------
\newpage
\LARGE Progress @ July 26\normalsize\vspace{3ex}

\par We aim to find a set of components $F = \{f_i\}_{i=1}^{K}$ which are fucntions in a RKHS $\mathcal{H}$ induced by a kernel function $k(\cdot, \cdot)$. Given $k(\cdot, \cdot)$ and the input space $\mathcal{X} = \mathbb{R}^D$, a reproducing kernel feature map \cite{scholkopf2002learning} $\phi: \mathcal{X}\rightarrow\mathbb{R}^\mathcal{X}$ can be defined as $\phi(\mathbf{x}) = k(\mathbf{x}, \cdot)$ and the inner product $\phi(\mathbf{x})^{T}\phi(\mathbf{y})$ can be computed as $k(\mathbf{x}, \mathbf{y})$ via the kernel trick.
Firstly, we define the kernel neural network problem

\begin{equation}
\min\sum_{n=1}^{N}\|y_n - t_n\|^2,
\end{equation}

where $t_n$ is the lable of the sample $\mathbf{x}_n$, and $y_n$ is the output of the kernel neural network with input sample $\mathbf{x}_n$. Then we talk the computation of $y_n$ in parametric form and non-parametric form separately.

\subsubsection{Parametric Representation}
Under the parametric representation, the functions $f$ take a parametric form which is independent of data: $\{f|f = \phi(\mathbf{a}), \mathbf{a}\in \mathbb{R}^D\}$. So we can compute $f(\mathbf{x})$ as $f(\mathbf{x}) = \phi(\mathbf{a})^T \phi(\mathbf{x}) = k(\mathbf{a}, \mathbf{x})$.\\

For single-layer neural network we have:

\begin{equation}
\phi(\mathbf{x}) = k(\mathbf{x},\cdot)
\end{equation}

\begin{equation}
f_i = \phi(\mathbf{a}_i) = k(\mathbf{a}_i,\cdot)
\end{equation}

\begin{equation}
f_i(\mathbf{x}) = f_i(\cdot)\phi(\mathbf{x}) = k(\mathbf{a}_i,\cdot) k(\mathbf{x},\cdot) = k(\mathbf{a}_i,\mathbf{x})
\end{equation}

\begin{equation}
y = \sum\limits_{i=1}^K w_i f_i(\mathbf{x}) = \sum\limits_{i=1}^K w_i k(\mathbf{a}_i,\mathbf{x})
\end{equation}

And for multi-layer neuarl network we have:

\begin{equation}
\phi^{(2)}(\mathbf{x}^{(2)}) = k^{(2)}(\mathbf{x}^{(2)},\cdot)
\end{equation}

\begin{equation}
f_i^{(2)}  = \phi^{(2)}(\mathbf{a}_i^{(2)}) = k^{(2)} (\mathbf{a}_i^{(2)},\cdot)
\end{equation}

\begin{equation}
f_i^{(2)}(\mathbf{x}^{(2)}) = f^{(2)}(\cdot)\phi^{(2)}(\mathbf{x}^{(2)}) = k^{(2)}(\mathbf{a}_i^{(2)},\cdot) k^{(2)}(\mathbf{x}^{(2)},\cdot) = k^{(2)}(\mathbf{a}_i^{(2)},\mathbf{x}^{(2)})
\end{equation}

\begin{equation}
y = \sum\limits_{i=1}^K w_i^{(2)} f_i^{(2)}(\mathbf{x}^{(2)}) = \sum\limits_{i=1}^K w_i^{(2)} k^{(2)}(\mathbf{a}_i^{(2)},\mathbf{x}^{(2)})
\end{equation}

\begin{equation}
\mathbf{x}^{(2)} = \left[ \begin{array}{c}
  k(\mathbf{a}_1,\mathbf{x})\\
  .\\
  .\\
  .\\
  k(\mathbf{a}_K,\mathbf{x})\\
\end{array}
  \right]
\end{equation}


\subsubsection{Non-parametric Representation}
Under the non-parametric representation, the functions $f$ admit a non-parametric form which is data dependent: $\{f|f = \sum_{i=1}^{N} \alpha_i k(\mathbf{x}_i, \cdot), \alpha_i\in\mathbb{R}\}$. So we can compute $f(\mathbf{x})$ as $f(\mathbf{x}) = \sum_{i=1}^{N}\alpha_i k(\mathbf{x}_i, \mathbf{x})$.\\

For single-layer neural network we have:

\begin{equation}
\phi(\mathbf{x}) = k(\mathbf{x},\cdot)
\end{equation}

\begin{equation}
f_i = \sum\limits_{i=1}^N \alpha_{ji} k(\mathbf{x}_i,\cdot)
\end{equation}

\begin{equation}
f_i(\mathbf{x}) = f_i(\cdot)\phi(\mathbf{x}) = \sum\limits_{i=1}^N \alpha_{ji} k(\mathbf{x}_i,\cdot) k(\mathbf{x}, \cdot) = \sum\limits_{i=1}^N \alpha_{ji} k(\mathbf{x}_i,\mathbf{x})
\end{equation}

\begin{equation}
y = \sum\limits_{i=1}^K w_i f_i(\mathbf{x}) = \sum\limits_{i=1}^K w_i \sum\limits_{i=1}^N \alpha_{ji} k(\mathbf{x}_i,\mathbf{x})
\end{equation}

And for multi-layer neural network we have:

\begin{equation}
\phi^{(2)}(\mathbf{x}^{(2)}) = k^{(2)}(\mathbf{x}^{(2)},\cdot)
\end{equation}

\begin{equation}
f_i^{(2)} = \sum\limits_{i=1}^N \alpha_{ji}^{(2)} k^{(2)}(\mathbf{x}_i^{(2)},\cdot)
\end{equation}

\begin{equation}
f_i^{(2)}(\mathbf{x}^{(2)}) = f_i^{(2)}(\cdot)\phi^{(2)}(\mathbf{x}^{(2)}) = \sum\limits_{i=1}^N \alpha_{ji}^{(2)} k^{(2)}(\mathbf{x}_i^{(2)},\cdot) k^{(2)}(\mathbf{x}^{(2)}, \cdot) = \sum\limits_{i=1}^N \alpha_{ji}^{(2)} k^{(2)}(\mathbf{x}_i^{(2)},\mathbf{x}^{(2)})
\end{equation}

\begin{equation}
y = \sum\limits_{i=1}^K w_i^{(2)} f_i^{(2)}(\mathbf{x}^{(2)}) = \sum\limits_{i=1}^K w_i^{(2)} \sum\limits_{i=1}^N \alpha_{ji}^{(2)} k^{(2)}(\mathbf{x}_i^{(2)},\mathbf{x}^{(2)})
\end{equation}

\begin{equation}
\mathbf{x}^{(2)} = \left[ \begin{array}{c}
  \sum\limits_{i=1}^N \alpha_{1i} k(\mathbf{x}_1,\mathbf{x})\\
  .\\
  .\\
  .\\
  \sum\limits_{i=1}^N \alpha_{Ki} k(\mathbf{x}_K,\mathbf{x})\\
\end{array}
  \right]
\end{equation}



%----------------------------------@ July 28---------------------------------------
\newpage
\LARGE Progress @ July 28\normalsize\vspace{3ex}

\par Firstly, we define the kernel neural network problem:
\color[rgb]{1,0,0}
\begin{equation}
  \min_{\{\mathbf{F}^{(i)} \subseteq{\mathcal{H}}\}_{i=1}^{m}} \sum_{n=1}^{N} \ell ( \{ f_{i^{(m)}}^{(m)} ( \{ f_{i^{(m-1)}}^{(m-1)} ( \cdots ( \{ f_{i^{(1)}}^{(1)}(\mathbf{x}_n) \}_{i^{(1)}=1}^{K^{(1)}} ) \cdots ) \}_{i^{(m-1)}=1}^{K^{(m-1)}} )\}_{i^{(m)}=1}^{K^{(m)}} )
  - \lambda\Omega(\{\mathbf{F}^{(i)} \}_{i=1}^{m})
\end{equation}
\color[rgb]{0,0,0}

\par \textbf{Parametric representation form}
\begin{equation}
\begin{split}
  \min_{\{\mathbf{A}^{(i)}\}_{i=1}^{m}} \sum_{n=1}^{N} \ell ( \{ k^{(m)} (\mathbf{a}_{i^{(m)}}, \{ k^{(m-1)} (\mathbf{a}_{i^{(m-1)}}, \cdots \{ k^{(1)}(\mathbf{a}_{i^{(1)}}, \mathbf{x}_n) \}_{i^{(1)}=1}^{K^{(1)}} \cdots ) \}_{i^{(m-1)}=1}^{K^{(m-1)}} )\}_{i^{(m)}=1}^{K^{(m)}} )\\
  - \lambda\Omega(\{\mathbf{A}^{(i)} \}_{i=1}^{m}, \{k^{(j)} \}_{j=1}^{m})
\end{split}
\end{equation}
where $\mathbf{A}=\{\mathbf{a}_i\}_{i=1}^{K}$\\

\par \textbf{Non-parametric representation form}
\begin{equation}
  \min_{\{\mathbf{r}^{(i)}\}_{i=1}^{m}} \sum_{n=1}^{N} \ell ( \{ \sum_{j^{(m)}=1}^{n} \alpha^{m}_{i^{(m)}j^{(m)}} k^{(m)} (\mathbf{x}_{j^{(m)}}^{(m)}, \mathbf{x}_{n}^{(m)})\}_{i^{(m)}=1}^{K^{(m)}} )
  - \lambda\Omega(\{\mathbf{r}^{(i)} \}_{i=1}^{m}, \{k^{(j)} \}_{j=1}^{m})
\end{equation}
where $\mathbf{r}=\{\bm{\alpha}_i\}_{i=1}^{K}$ and:

\begin{equation}
\begin{split}
\mathbf{x}_{j^{(m)}}^{(m)} &= \{  \sum_{j^{(m-1)}=1}^{N} \bm{\alpha}_{i^{(m-1)} j^{(m-1)}}^{(m-1)} k^{(m-1)} (x_{j^{(m-1)}}^{(m-1)}, x_{j^{(m)}}^{(m-1)}) \}_{i^{(m-1)}=1}^{k^{(m-1)}}\\
&\cdot\\
&\cdot\\
&\cdot\\
\mathbf{x}_{j^{(2)}}^{(2)} &= \{  \sum_{j^{(1)}=1}^{N} \bm{\alpha}_{i^{(1)} j^{(1)}}^{(1)} k^{(1)} (x_{j^{(1)}}^{(1)}, x_{j^{(2)}}^{(1)}) \}_{i^{(1)}=1}^{k^{(1)}}
\end{split}
\end{equation}



\newpage
\bibliographystyle{ieeetr}
\bibliography{mybibtex}

\end{document}